\documentclass[onecolumn, 12pt]{IEEEConf}
\usepackage{xr}
\usepackage{zref}
\usepackage{graphicx}
\usepackage{float}
\usepackage[export]{adjustbox}
\usepackage{setspace}
\usepackage{url}


\input{"C:/Users/Klaethis/Documents/Latex Templates/CodeFormat.tex"}
\input{"C:/Users/Klaethis/Documents/Latex Templates/MathFormat.tex"}


\renewcommand{\thesection}{\Roman{section}}
\renewcommand{\thesubsection}{\Alph{subsection}}
\renewcommand{\thesubsubsection}{\roman{subsubsection}}


\title{Final Project Report}
\author{Mike Zimmerman \\ ECE-3780-001 \\ Communications Systems}
\date{\today}


\begin{document}
    \maketitle
    \doublespacing
    \section{Introduction}
        The purpose of this project was to demonstrate a wireless communication system.
        The system chosen for this project was the Wi-Fi\footnote{Wi-Fi is a registered trademark of the Wi-Fi alliance.} wireless communication protocol.
        This protocol was selected for its ubiquity and ease of use.
        In order to implement our communication, a microcontroller (MCU), was selected that natively supported Wi-Fi.

        In order to show that wireless communication was possible, two different systems were created.
        One as the server, and another as the client.
        The server has an environment sensor attached, and would send the data to connected clients.
        The client would detect the server device, connect, gather the transmitted data, and display the values on a screen.
        As a secondary client, the server would also produce a simple webpage that a computer could display.

    \section{Wi-Fi Background}
        The chosen wireless communication for this project was Wi-Fi.
        It was created as it is known today by the Wi-Fi Alliance in september 1998, and is now the most popular wireless communication protocol.\cite{Wi-Fi_History}
        It uses the IEEE 802.11 standard. 

        Wi-Fi uses radio waves in the 2.4 GHz or 5 GHz frequencies to transmit and receive digital data.
        These frequencies are in Ultra High, and Super High frequency bands.

        % Talk more about how wifi packets are built and transmitted.

    \section{ESP8266}
        % Talk about Espressif's general history

        % Talk about the initial development of the ESP8266 as a assistant controller.

        % Talk about the development of the individual module.

    \section{Final Project}
        % Short intro to the idea of the weather station

    \subsection{Server Development}
        % Talk about the server circuit

        % Talk about the server code

        % Talk about the ability for the server to be updated over the air

    \subsection{Client Development}
        % Talk about the client circuit

        % Talk about the client code

    \subsection{Further Development}
        % Talk about the development of the computer application

        % Talk about how many clients can connect at a time. (4 currently, but it might be possible to extend that)

    \section{Results}
        % Talk about how stable the connection is

        % Talk about the distance the server can be accessed at 43'6"

    \section{Conclusion}

        % Wrap it up!

    \section{Appendix}

    \bibliographystyle{IEEEConf}
    \bibliography{Bibliography.bib}
\end{document}